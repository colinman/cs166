\documentclass[11pt,oneside,a4paper]{article}
% \renewcommand{\labelenumii}{(\alph{enumi})}
\usepackage{forest}
\usepackage{amsmath}
\usepackage{enumitem}
\usepackage{fancyhdr}
\pagestyle{fancy}

\begin{document}
\title{CS166 HW1}
\author{Colin Man (colinman@stanford.edu), Kenny Xu (kenxu95@stanford.edu)}
\lfoot{colinman@stanford.edu}

\maketitle

\begin{description}

\item[Problem 3]\noindent 

\begin{enumerate}[label=\roman*]
\item We can construct hybrid structures of depth $k-1$ by breaking them up into block sizes $b_1, b_2, b_3, \ldots, b_{k-1}$ and use a sparse table RMQ on each level to achieve $<O(n\log n),O(1)>$ for that level. The preprocessing time of the hybrid structure is then \[
    O(n + p(\frac{n}{b_1}) + (\frac{n}{b_1}) p(\frac{b_1}{b_2}) + (\frac{n}{b_1})(\frac{b_1}{b_2}) p(\frac{b_2}{b_3}) + \cdots + (\frac{n}{b_{k-2}}) p(\frac{b_{k-2}}{b_{k-1}}) + (\frac{n}{b_{k-1}}) p(b_{k-1}))
\] If we let \begin{align*}
               b_1=\log n \\
               b_2=\log b_1 \\
               \cdots \\
               b_{k-1} = \log b_{k-2}
             \end{align*}
Since $p(x)=x\log x$ using a sparse table RMQ, we can use the trick mentioned in lecture to reduce every term but the last as follows: \begin{align*}
  O((\frac{n}{b_{i-1}})p(\frac{b_{i-1}}{b_i})) &= O((\frac{n}{b_{i-1}})\frac{b_{i-1}}{b_i}\log(\frac{b_{i-1}}{b_i})) \\ &= O((\frac{n}{b_{i-1}})\frac{b_{i-1}}{\log b_{i-1}}\log b_{i-1}) \\ &= O((\frac{n}{b_{i-1}})b_{i-1}) \\ &= O(n)
\end{align*}
Upon reduction, we have a preprocessing time of: \begin{align*}
                           O(n + (\frac{n}{b_{k-1}}) (b_{k-1})\log b_{k-1}) &= O(n\log b_{k-1}) \\ &= O(n\log \log b_{k-2}) \\ &= O(n\log^{(k-1)} b_1) \\ &= O(n\log^{(k-1)} \log n) \\ &= O(n\log^{(k)} n)
                         \end{align*}

Lookup is performed similar to a two level hybrid query: at each of the k levels, we perform RMQ queries one level deeper on the blocks that contain the two boundary indices and perform RMQ on the blocks between them (at the current level). This gives a query time of $O(k) = O(1)$ since k is constant. \\

Thus, the time complexity of our hybrid structure is $<O(n\log^{(k)}n), O(1)>$ as required.

\item The increased query time arises from the fact that the query time is $O(k)$ as shown in part (i). Thus, as k increases, the runtime increases. However, since k is a constant, all the hybrid structures still have a query time of $O(1)$, which does not contradict our result in (i).

\end{enumerate}

\end{description}

\end{document}


%%% Local Variables:
%%% mode: latex
%%% TeX-master: t
%%% End:
