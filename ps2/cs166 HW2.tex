\documentclass{article}
\usepackage[utf8]{inputenc}
\usepackage[]{amsmath}
\usepackage{amsthm,amssymb}
\usepackage{enumerate}% http://ctan.org/pkg/enumerate
\usepackage{hyperref}
\usepackage{tikz}
\usepackage{ mathrsfs}
\usepackage[margin=0.5in]{geometry}

\usepackage{tikz}
\usepackage{pgfplots}

% no indents pls
\setlength\parindent{0pt}


%%% defs for theorem environments
\renewcommand{\qedsymbol}{\rule{0.7em}{0.7em}} % black box for QED
\newtheorem*{thm}{Theorem}
\newtheorem{lem}{Lemma}
%defining style for Cases
\newtheoremstyle{casestyle}{\topsep}{1pt}{}{0pt}{\itshape}{.}{5pt plus 1pt minus 1pt}{}
\theoremstyle{casestyle}
\newtheorem{case}{Case}
%smallcaps-ing "Proof" header
\let\oldproofname=\proofname
\renewcommand{\proofname}{\rm\sc{\oldproofname}}

\begin{document}

\section *{Problem 2}
\begin{enumerate}[i]
    \item 
    Because there is only one pattern string, our automaton will be in the shape of a line. (Think linked-list)

    \item 
    We need trie edges because we still need to advance character by character. We still need suffix links because suffixes can still be preserved within the same string on a mismatch. For example, if we look for "ababe" in "abababab" we will match up to index $4$ and fail when 'e' is compared with 'a'. Instead of falling back to index $0$, we can fall back to index $2$ and preserve the suffix "ab" so we don't miss anything.\\
    
    However, we don't need output links. Output links only fix the case where there are multiple pattern strings that are substrings of each other. Because we only deal with a single pattern string, this is not a problem.\\
    
    \item 
    See code.
    
    \item 
    Let $P$ be the pattern string. We chose to represent the matching automaton as an array of integers $A$ with a length $1$ greater than $length(P)$. This way, each index in $A$ corresponds to a node in the automaton. The associated character of every trie link can be accessed by indexing into $P$ based on your current index in $A$. This makes moving to the next node along the trie links very easy - just increment your current index in $A$. The values in $A$ correspond to suffix links. That is, the value of $A[i]$ is the index of the node pointed to be node $i$'s suffix pointer (this also means that $A[i] < i$). Therefore, by utilizing $P$ and $A$, we can access all the necessary information to perform string matching.\\
    
    
\end{enumerate}


\section *{Problem 3}
Let $LCE_{S_1, S_2}(i, j)$ be the length of the longest common prefix between any two strings $S_1$ and $S_2$ starting at positions $i$ and $j$. Our solution to the k-approximate matching problem will utilize the Longest Common Extension solution present in lecture.\\

Our solution for the k-approximate matching problem, where $T$ is the text string and $P$ is the query string:\\

Keep an error count $C$. For every character position $i$ in $T$, calculate $L_1 = LCE_{T, P}(i, 0)$ using the algorithm presented in lecture. If $L_1 < length(P)$, increment the error count $C$ and calculate $L_2 = LCE_{T, P}(i + L_1 + 1, L_1 + 1)$. Note that this essentially skips the mismatched character, utilizing one of our $k$-allowed mismatch errors. If $C + L_1 + L_2$ is still $< length(P)$, skip the mismatched character and calculate the next $LCE$ and so on. Repeat this process until either:\\
$1)$ The total length of the extensions $(C + L_1 + L_2 + ...) >= length(P)$. Return true because we have found a appropriate solution.\\
$2)$ The error count $C$ exceeds the allowed $k$ or we run off the end of $T$ before $>= length(P)$ consecutive characters have been matched. Then, move on to the next character position $i$. If every character position $i$ in $T$ is evaluated without success, we return false.\\

\begin{proof} 
We will first prove the correctness of this algorithm. Let $T[i:j]$ be the notation for the substring of $T$ starting at index $i$ and ending at $j$ (for convenience, if $j > length(T)$ just stop the substring at the end of $T$).\\

For a given character index $i$ in $T$, if $T[i:i+length(P)]$ is the correct match for $P$, then $T[i:i+length(P)]$ will match exactly with $P$ with at most $k$ character mismatches. We do exactly this with our $LCE$ construction. By definition, $LCE$ counts the number of consecutive matching characters until a mismatch occurs. And when a mismatch occurs, we simply increment our error count and skip past the offending mismatch, rerunning $LCE$ starting at the new index. Therefore, we are guaranteed that if $T[i:i+length(P)]$ is a correct match for $P$, the sum of the lengths returned by our $LCE$ calls and the mismatches we skipped over (tallied in our error count) will be $>= length(P)$, which is our success condition. On the other hand, we are also guaranteed that if $T[i:i+length(P)]$ is not a correct match of $P$, the error count will exceed the allowed $k$ (or we run off the end of $T$) before we finish matching the entire string, which is our failure condition.\\

Because this verification starts at every index $i$ in $T$, if a solution exists, we are guaranteed to find it. And if a solution does not exist, we will fail at every $i$ and eventually return false. Therefore, our algorithm will yield the correct answer.
\end{proof}

\begin{proof}
We will now prove the runtime of this algorithm. In order to run $LCE$, we must preprocess the strings $T$ and $P$, which takes linear time: $O(m + n)$. However, $LCE$ itself runs in constant time. When we run our algorithm, we look at $O(m)$ starting indexes in $T$, and evaluate each potential matching substring in $O(k)$. It must be $O(k)$ because $LCE$ runs in $O(1)$ time and we can run $LCE$ a maximum of $k$ times before we exceed the allowed error count. Therefore, running the entire algorithm will take $O(mk)$. Adding in preprocessing, we get a final runtime of $O(mk + n)$.
\end{proof}

\end{document}


